\documentclass[10pt,serif]{beamer}
\usepackage{latexsym}
\usepackage{amsmath}
%\usepackage{mathspec}
\usepackage{amsthm}
\usepackage{amssymb}
% \usepackage{textcomp} 
% \usepackage{ulem}
% \usepackage{fontspec}     
\usepackage[british]{babel}
\usepackage{enumitem}
% \usepackage{pifont}
\usepackage{geometry}
\usepackage{hyperref}
\usepackage[table]{xcolor}
\usepackage{tikz}
\usepackage{colortbl}
% \usepackage{mathrsfs}

% \usepackage{eucal}
\usepackage{dsfont}
% \usepackage[most]{tcolorbox}
% \usepackage{mathdots}
\usepackage{minted} %? used for code-integration
% \usepackage{titling}
\usepackage[citestyle=alphabetic,bibstyle=alphabetic]{biblatex}
\definecolor{darkmagenta}{rgb}{.38,0,0.38}
\hypersetup{
    colorlinks,
    citecolor=darkmagenta,
    filecolor=darkmagenta,
    linkcolor=darkmagenta,
    urlcolor=darkmagenta
} %? Change hyperlink's styles

%%%%%%%%%%%%%%? courbes
\usepackage{amsfonts,amscd}
\usepackage{pgfplots}
%%%%%%%%%%%%%?

%%%%%%%%%%%%%
\DeclareMathOperator{\mat}{Mat}
\DeclareMathOperator{\card}{Card}
\DeclareMathOperator{\len}{len}
\DeclareMathOperator{\erf}{erf}
%%%%%%%%%%%%%


%%%%%%%%%%%%%
\newcommand{\dt}{\mathrm{d}}
\newcommand{\un}{\mathds1}
\renewcommand{\o}{\scriptstyle\mathcal{O}}
\renewcommand{\O}{\mathcal{O}}
\renewcommand{\emptyset}{\varnothing}
\newcommand{\lbd}{\lambda}
\renewcommand{\phi}{\varphi}
\def\inte #1 #2 { [\![#1,#2]\!] }
%%%%%%%%%%%%%


%%%%%%%%%%%%%
\newcommand{\N}{\mathbb{N}}
\newcommand{\R}{\mathbb{R}}
\newcommand{\Z}{\mathbb{Z}}
\newcommand{\C}{\mathbb{C}}
\newcommand{\K}{\mathbb{K}}
\newcommand{\Q}{\mathbb{Q}}
\renewcommand{\P}{\mathbb{P}}
\newcommand{\M}{\mathcal{M}}
\renewcommand{\r}{\mathcal{R}}
\renewcommand{\S}{\mathcal{S}}
%%%%%%%%%%%%%


\renewcommand{\arraystretch}{1.4} %? For better matrix
\renewcommand{\thesection}{\arabic{section}} %? Roman number for sections

% \theoremstyle{definition}
% \newtheorem{definition}{Definition}[subsubsection]

% \theoremstyle{theorem}
% \newtheorem{theorem}{Th\'eor\`eme}[subsubsection]

% \theoremstyle{lemma}
% \newtheorem{lemma}[theorem]{Lemme}

% \theoremstyle{corollary}
% \newtheorem{corollary}{Corollaire}[theorem]

% \theoremstyle{remark}
% \newtheorem*{remark}{Remarque}


\newenvironment{code}{
    \VerbatimEnvironment
    \begin{minted}[mathescape,
        linenos,
        numbersep=5pt,
        gobble=2,
        frame=lines,
        framesep=2mm]{cpp}%
}{
        \end{minted}%
}


% \flushbottom




\usetheme{CambridgeUS}
\usecolortheme{rose}
\setbeamercolor{frametitle}{fg=darkmagenta}
\setbeamercolor{title}{fg=darkmagenta!80!white}

\setbeamercolor{palette primary}{fg=black, bg=gray!30!white}
\setbeamercolor{palette secondary}{fg=black, bg=gray!20!white}
\setbeamercolor{palette tertiary}{fg=white, bg=darkmagenta}
\setbeamercolor{palette quaternary}{bg=darkmagenta, fg=white}

% \setbeamersize{
%   text margin left = 0.25cm, % normalement c'est 1 cm
%   text margin right = 0.25cm % normalement c'est 1 cm
% }



%% Pour supprimer la barre de navigation
\beamertemplatenavigationsymbolsempty

% \newlength{\hauteurFrameTOC}
% \setlength{\hauteurFrameTOC}{0.65\textheight}
% \newlength{\upTOC}
% \setlength{\upTOC}{-10pt}


% \AtBeginSection[]
% {
%   \begin{frame}<beamer>
%     \frametitle{Sommaire}
%     \vspace{\upTOC}
%     \hfill
%     \parbox[t]{.95\textwidth}{
%       \begin{minipage}[c][\hauteurFrameTOC]{\textwidth}
%         \tableofcontents[currentsection]
%       \end{minipage}
%     }
%   \end{frame}
% }


% Pour changer le stye des points itemize
% \setbeamertemplate{itemize item}[triangle]

% \newlength{\lengthleft}
% \newlength{\lengthright}


% \setbeamertemplate{headline}[default]
% \setbeamertemplate{footline}[default]

% \defbeamertemplate*{footline}{infolines theme}
% {
%   \hbox{%
%     \begin{beamercolorbox}[wd=.22\paperwidth,ht=2.25ex,dp=1ex,center]{author in head/foot}%
%       TIPE 2021
%     \end{beamercolorbox}%
%     \begin{beamercolorbox}[wd=.70\paperwidth,ht=2.25ex,dp=1ex,center]{title in head/foot}%
%       \insertshorttitle
%     \end{beamercolorbox}%
%     \begin{beamercolorbox}[wd=.08\paperwidth,ht=2.25ex,dp=1ex,center]{author in head/foot}%
%       \insertframenumber/\inserttotalframenumber
%   \end{beamercolorbox}  }%
% }

% Supprime les ombres des block (ces trucs buguent aujourd'hui)
\setbeamertemplate{blocks}[rounded][shadow=false]


% Pour cacher des frames:
% \newcounter{FrameNumberBeforeAppendix}
% \newenvironment{annexes}{
%   \setcounter{FrameNumberBeforeAppendix}{\value{framenumber}}
% }{
%   \setcounter{framenumber}{\value{FrameNumberBeforeAppendix}}
%%}



\usepackage{subcaption}

\def\bibfont{\footnotesize}
% \renewcommand\bibliographytypesize{\small}
{\footnotesize
\bibliography{./report.bib}}
\usepackage{wrapfig}
\usepackage{ragged2e}
\usepackage{tabularx}
\usepackage{tkz-base}
\usepackage{tikz}
\usepackage{csquotes}
\usepackage{pgfplots}
\usepackage{datetime2}
\usepackage{adjustbox}
\usepackage{cleveref}
\usepackage{multicol}
\usepackage{graphicx}
\usepackage{multirow}
\usepackage{tabularx}
\usepackage{multimedia}



\newcolumntype{C}{>{\centering\arraybackslash}X}
\newcolumntype{s}{>{\hsize=.3\hsize\linewidth=\hsize}C}
\newcolumntype{D}{>{\hsize=.4\hsize\linewidth=\hsize}C} %Double width column


\AtBeginSection[]
{
  \begin{frame}
    \frametitle{Table of content}
    \tableofcontents[currentsection]
  \end{frame}
}

\uselanguage{English}
\languagepath{English}

\pgfplotsset{compat=1.17}
\renewcommand{\today}{\number\day\textsuperscript{\scriptsize\text{th}} \DTMenglishmonthname{\month} \number\year}
% \useoutertheme{smoothbars}

% \setbeamertemplate{title page}[default][colsep=-0bp,rounded=false]
% \setbeamertemplate{frametitle}[default][colsep=-4bp,rounded=false,shadow=false]
% \setbeamertemplate{blocks}[rounded][shadow=false]
% \setbeamertemplate{headline}[shadow=false]
% \setbeamertemplate{subsection in head}[shadow=false]
% \setbeamertemplate{section in head}[shadow=false]
% \setbeamertemplate{beamercolorbox}[shadow=false]

\makeatletter
\def\@makefnmark{}
\makeatletter

% \setbeamertemplate{footnote}{%
%   \parindent 1em\noindent
%   \raggedright
%   \insertfootnotetext\par
% }

%%%%%%%%%%%%%%%%%%%%%%%%%%%%%%%%%%%%%%%%%%%%%%%%

% \title{
%   Arbitrages statistiques dans l'apprentissage automatique confidentiel
% }
% \author{{\sc Alexi Canesse}\\ Candidat 557}
% \date{\today}

\title[] %optional
{EEG and MEG Data Preprocessing}

\subtitle{Oral presentation}

\author[Esteban Christiann, Alexi Canesse]{{Alexi \sc Canesse} and {Esteban \sc Christiann},\\ ENS Paris-Saclay, France}

\institute[] % (optional)
{
    Project for the time series course\\
    Part of the MVA program at ENS Paris-Saclay.\\
}

\date[Oral presentation] % (optional)
{\today}

% \logo{\includegraphics[height=1cm]{}}

%%%%%%%%%%%%%%%%%%%%%%%%%%%%%%%%%%%%%%%%%%%%%%%%

% \pgfplotsset{every axis/.append style={
%         scaled y ticks = false,
%         scaled x ticks = false,
%         y tick label style={/pgf/number format/.cd, fixed,
%                             int detect,1000 sep={\;},precision=3},
%         x tick label style={/pgf/number format/.cd, fixed, fixed zerofill,
%                             int detect, 1000 sep={},precision=3}
%     }
% }

\setlength{\parskip}{1em}

\begin{document}
% \addbibresource{pres.bib}


\begin{frame}
  \titlepage
\end{frame}

\section[\color{white} Introduction]{Introduction}\label{sec:introduction}

\begin{frame}{Introduction}

\end{frame}

\begin{frame}{Contributions}

\end{frame}

\section[\color{white} Methods]{Methods}\label{sec:methods}

\begin{frame}{Robust detrending}
  Usual detrending methods are sensible to artefacts.

  $\implies$ Robust detrending: estimate the trend ignoring outliers, and flag high errors as outliers for the next iteration

% \begin{pseudocode}
% detrending(signal):
%     weights = [1, 1, |\(\dots\)|, 1] # weights for outlier detection (0 means outlier)
%     iterrate |\(n_\text{iter}\)| times:
%         projected_signal = fit_to_basis_using_weights(signal, weights)
%         error_on_projection = abs(signal - projected_signal)
%         for t in len(signal):
%             if error_on_projection[t]/std(error_on_projection) > threshold:
%                 weights[t] = 0
%             else:
%                 weights[t] = 1
%     return signal - fitted_signal # detrended signal
% \end{pseudocode}
\end{frame}

\begin{frame}{Inpainting}

  Inpainting is the process of reconstructing the signal at timesteps affected by glitches. Here, we assume that glitches locations are known.

  This algorithm first estimates the linear relationship between channels and use it to reconstruct the signal.

  % \begin{pseudocode}
  %   inpaint(x, w):
  %       N = Number of channels
  %       for n in range(N):
  %           Partition the time axis into |\(K\)| subsets based on validity of channels
  %           for k in range(K):
  %               T_k = timesteps in subset k
  %               Tprime = subset of T_k where ch. n is valid
  %               Tinpaint = subset of T_k where ch. n has to be reconstructed
  %               Estimate the linear relationship between ch. n and valid ch. on Tprime
  %               Use the relationship to reconstruct ch. n on Tinpaint
  %       return new_x
  %   \end{pseudocode}

\end{frame}

\begin{frame}{Outlier detection}

  To flag outliers, we use the previous algorithm and flag poorly reconstructed values as outliers for the next iteration.

% \begin{pseudocode}
% outlier_detection(x, thres=2., maxiter=20):
%     w = ones_like(x) # Initially assume there are no outliers
%     for it in range(maxiter):
%         xbar = inpaint(x, w) # Try to reconstruct the data
%         d = np.abs(x - xbar)
%         w = d < thres * d.std() # Flag high reconstruction errors as outliers
%     return w
% \end{pseudocode}

\end{frame}

\begin{frame}{Robust rereferencing}

% \begin{pseudocode}
% rereferencing(signal):
%     weights = 1 - outlier_detection(signal) # 1 : valid data / 0 : outlier
%     weighted_mean = mean_using_weights(signal, weights)
%     return signal - weighted_mean
% \end{pseudocode}

\end{frame}

\begin{frame}{Step removal}

%   \begin{pseudocode}
% # Recursively look for steps in the signal
% step_detection(signal, depth)
%     |\(\forall (t,T)\)| M[t][T] = mean(x[t:T+1])
%     |\(\forall (t,T)\)| V[t][T] = |\(\sum_{i=t}^T (x_i - M_t^T)^2\)|
    
%     t0 = argmin(V[1][t] + V[t][T], t) # most likely position to be a step
%     if not(is_step(signal, t0)): # Check if t0 is indeed the position of a step
%         return [] # no steps
    
%     # Check for steps on both sides of t0
%     steps_left = step_detection(signal[:t0], depth - 1)
%     steps_right = t0 + step_detection(signal[t0:], depth - 1) # + t0 because signal si shifted

%     return concatenate(steps_left, [t0], steps_right)

% step_removal(signal, depth):
%     steps = step_detection(signal, depth)

%     for step in steps:
%         signal[steps:] -= difference_mean_before_and_after_step(signal, step)

%     return signal # Without steps

% \end{pseudocode}

\end{frame}

\begin{frame}{Ringing removal}

  Ringing artefacts are caused by the antialiasing filter. First, we estimate the parameters of this filter to model its impulse response and cancel it, which removes ringing artefacts.

%   \begin{pseudocode}
% ringing_removal(x, step_list):
%     for n in range(number of channels):
%         for step in step_list[n]:
%             Estimate filter parameters on x[step : step + 100, n]
%             imp_res = impulse response of the filter
%             x[step : step+100, n] -= imp_res
% \end{pseudocode}

\end{frame}

\section[\color{white} Data]{Data}\label{sec:data}

\begin{frame}{Data}

\end{frame}

\section[\color{white} Data]{Results}\label{sec:results}

\begin{frame}{Results}

\end{frame}


\end{document}
