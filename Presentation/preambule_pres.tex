\usepackage{latexsym}
\usepackage{amsmath}
%\usepackage{mathspec}
\usepackage{amsthm}
\usepackage{amssymb}
% \usepackage{textcomp} 
% \usepackage{ulem}
% \usepackage{fontspec}     
\usepackage[british]{babel}
\usepackage{enumitem}
% \usepackage{pifont}
\usepackage{geometry}
\usepackage{hyperref}
% \usepackage[table]{xcolor}
\usepackage{tikz}
\usepackage{colortbl}
% \usepackage{mathrsfs}

% \usepackage{eucal}
\usepackage{dsfont}
% \usepackage[most]{tcolorbox}
% \usepackage{mathdots}
\usepackage{minted} %? used for code-integration
% \usepackage{titling}
\usepackage[citestyle=alphabetic,bibstyle=alphabetic]{biblatex}
\definecolor{darkmagenta}{rgb}{.38,0,0.38}
\hypersetup{
    colorlinks,
    citecolor=darkmagenta,
    filecolor=darkmagenta,
    linkcolor=darkmagenta,
    urlcolor=darkmagenta
} %? Change hyperlink's styles

%%%%%%%%%%%%%%? courbes
\usepackage{amsfonts,amscd}
\usepackage{pgfplots}
%%%%%%%%%%%%%?

%%%%%%%%%%%%%
\DeclareMathOperator{\mat}{Mat}
\DeclareMathOperator{\card}{Card}
\DeclareMathOperator{\len}{len}
\DeclareMathOperator{\erf}{erf}
%%%%%%%%%%%%%


%%%%%%%%%%%%%
\newcommand{\dt}{\mathrm{d}}
\newcommand{\un}{\mathds1}
\renewcommand{\o}{\scriptstyle\mathcal{O}}
\renewcommand{\O}{\mathcal{O}}
\renewcommand{\emptyset}{\varnothing}
\newcommand{\lbd}{\lambda}
\renewcommand{\phi}{\varphi}
\def\inte #1 #2 { [\![#1,#2]\!] }
%%%%%%%%%%%%%


%%%%%%%%%%%%%
\newcommand{\N}{\mathbb{N}}
\newcommand{\R}{\mathbb{R}}
\newcommand{\Z}{\mathbb{Z}}
\newcommand{\C}{\mathbb{C}}
\newcommand{\K}{\mathbb{K}}
\newcommand{\Q}{\mathbb{Q}}
\renewcommand{\P}{\mathbb{P}}
\newcommand{\M}{\mathcal{M}}
\renewcommand{\r}{\mathcal{R}}
\renewcommand{\S}{\mathcal{S}}
%%%%%%%%%%%%%


\renewcommand{\arraystretch}{1.4} %? For better matrix
\renewcommand{\thesection}{\arabic{section}} %? Roman number for sections

% \theoremstyle{definition}
% \newtheorem{definition}{Definition}[subsubsection]

% \theoremstyle{theorem}
% \newtheorem{theorem}{Th\'eor\`eme}[subsubsection]

% \theoremstyle{lemma}
% \newtheorem{lemma}[theorem]{Lemme}

% \theoremstyle{corollary}
% \newtheorem{corollary}{Corollaire}[theorem]

% \theoremstyle{remark}
% \newtheorem*{remark}{Remarque}


\newenvironment{code}{
    \VerbatimEnvironment
    \begin{minted}[mathescape,
        linenos,
        numbersep=5pt,
        gobble=2,
        frame=lines,
        framesep=2mm]{cpp}%
}{
        \end{minted}%
}

\newenvironment{pseudocode}{
    \VerbatimEnvironment
    \begin{minted}[mathescape,
        linenos,
        numbersep=5pt,
        gobble=2,
        frame=lines,
        escapeinside=||,
        framesep=2mm,
        rulecolor=magenta]{python}%
}{
    \end{minted}%
}



% \flushbottom




\usetheme{CambridgeUS}
\usecolortheme{rose}
\setbeamercolor{frametitle}{fg=darkmagenta}
\setbeamercolor{title}{fg=darkmagenta!80!white}

\setbeamercolor{palette primary}{fg=black, bg=gray!30!white}
\setbeamercolor{palette secondary}{fg=black, bg=gray!20!white}
\setbeamercolor{palette tertiary}{fg=white, bg=darkmagenta}
\setbeamercolor{palette quaternary}{bg=darkmagenta, fg=white}

% \setbeamersize{
%   text margin left = 0.25cm, % normalement c'est 1 cm
%   text margin right = 0.25cm % normalement c'est 1 cm
% }



%% Pour supprimer la barre de navigation
\beamertemplatenavigationsymbolsempty

% \newlength{\hauteurFrameTOC}
% \setlength{\hauteurFrameTOC}{0.65\textheight}
% \newlength{\upTOC}
% \setlength{\upTOC}{-10pt}


% \AtBeginSection[]
% {
%   \begin{frame}<beamer>
%     \frametitle{Sommaire}
%     \vspace{\upTOC}
%     \hfill
%     \parbox[t]{.95\textwidth}{
%       \begin{minipage}[c][\hauteurFrameTOC]{\textwidth}
%         \tableofcontents[currentsection]
%       \end{minipage}
%     }
%   \end{frame}
% }


% Pour changer le stye des points itemize
% \setbeamertemplate{itemize item}[triangle]

% \newlength{\lengthleft}
% \newlength{\lengthright}


% \setbeamertemplate{headline}[default]
% \setbeamertemplate{footline}[default]

% \defbeamertemplate*{footline}{infolines theme}
% {
%   \hbox{%
%     \begin{beamercolorbox}[wd=.22\paperwidth,ht=2.25ex,dp=1ex,center]{author in head/foot}%
%       TIPE 2021
%     \end{beamercolorbox}%
%     \begin{beamercolorbox}[wd=.70\paperwidth,ht=2.25ex,dp=1ex,center]{title in head/foot}%
%       \insertshorttitle
%     \end{beamercolorbox}%
%     \begin{beamercolorbox}[wd=.08\paperwidth,ht=2.25ex,dp=1ex,center]{author in head/foot}%
%       \insertframenumber/\inserttotalframenumber
%   \end{beamercolorbox}  }%
% }

% Supprime les ombres des block (ces trucs buguent aujourd'hui)
\setbeamertemplate{blocks}[rounded][shadow=false]


% Pour cacher des frames:
% \newcounter{FrameNumberBeforeAppendix}
% \newenvironment{annexes}{
%   \setcounter{FrameNumberBeforeAppendix}{\value{framenumber}}
% }{
%   \setcounter{framenumber}{\value{FrameNumberBeforeAppendix}}
%%}

